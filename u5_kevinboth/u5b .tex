\documentclass [ ] {article}
\title {CS 102 \LaTeX ~\"Ubung}
\author {Kevin Both}

\begin{document}
\maketitle

\section{Das ist der erste Abschnitt}
Hier k\"konnte auch anderer Text stehen.
\section{Tabelle}
Unsere wichtigsten Daten finden Sie in Tabelle 1.
\begin{table}[h]
\begin{tabular}{c|c|c|c}
 \textbf{ }&\textbf{Punkte erhalten}&\textbf{Punkte m\"oglich}&\textbf{\%}\\
\hline{Aufgabe 1}& {2} &{4}&{0.5}\\
{Aufgabe 2} &{3} &{3} &{1}\\
{Aufgabe 3} &(3) &{3}&{1}
\end{tabular}
Caption {Diese Tabelle kann auch andere Werte beinhalten}
\end{table}
\section {Formeln}
\subsection {Pythagoras}
der Satz des Pythagoras errechnet sich wie folgt: $ a^{2} +b^{2}=c^{2}$. Daraus k\"onnen wir die L\"ange der Hypothenuse wie folgt berechnen: $c= \sqrt {a^{2}+b^{2}}$
\subsection{Summen}
Wir k\"innen auch die Formel f\"ur eine Summe angeben:\\
\begin {center}

$ s=\sum\limits_{i=1}^{n}=\frac{n*(n+1)}{2}$






\end{center}
\end{document}